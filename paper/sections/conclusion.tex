\section{Conclusion}
\label{sec:conclusion}
This study exhibits explorative character to firstly evaluate
the audibility of constant phase shifts.
%
For signals with rather complex structure, arbitrary constant phase shifts can only
be realized with digital signal processing.
%
Based on the Hilbert transform (i.e. the special case of $-90\degree$ constant
phase shift with unit magnitude), for which the infinite impulse responses
are well known in continuous-time and discrete-time signal domain, this paper
introduces the discrete-time infinite impulse response for an arbitrary constant
phase shifter.
%
For practical implementations an FIR design is proposed with special attention
to retain unit magnitude.
%
Furthermore, for periodic signals a periodic convolution is discussed.
%
Under the periodicity assumption, hereby the ideal phase shift filter without any
approximations or limitations can be applied.
%
The periodic convolution can be computed within DFT domain with high performance,
for which the spectrum of the constant phase shifter is given.
%
\NewL The results of the conducted listening experiment can be referred to the following
deductions.
%
Untrained, unconditioned listeners that have been repeatedly confronted with
multiple low frequency square wave bursts, lowpass filtered noise,
a transient castanet rhythm and a percussion sequence in a randomized sequence,
in general show different detection performances of applied constant phase shifts.
%
\NewL The majority of listeners was able to discriminate constant phase shifts
of $-90\degree$ and $-45\degree$ for square wave bursts with comparably little
demand and very high detection rate.
%
A 100\% detection rate was achieved by musical experts.
%
These findings are according to the known results with respect to other
low frequency group delay distortion of square waves.
%
\NewL For pink noise the majority of listeners is not able to detect constant
phase shift treatments of $-90\degree$.
%
However, the results indicate that by musical background and audio expertise
higher detection rates can be achieved, that might be tested for statistical
significance by a more sensitive test design.
%
An adapted effect size of about $g=0.2$ to $0.25$ seems to be reasonable for this.
%
The same observation and conclusion holds for the $-90\degree$ constant phase shift
of the castanets signal.
%
For both signals trained listeners are able to detect the treatment with statistical
significance.
%By intense training an effect size of about $g=0.4$, i.e. our
The initial guessed effect size can be well assumed for both
stimuli.
%
\NewL All listeners were not able to detect constant phase shift of $-90\degree$
for the sequence containing percussion material with full audio bandwidth.
%
Here, highest judgment demand was reported by qualitative statements, very
often admitting pure guessing.
%
The comparably longest rating durations might reflect this fact as well.
%
This insensitivity might be due to complex spectrum and full audio bandwidth,
compared to the other used audio contents.
%
An adapted effect size of about $g=0.15$ seems to be an appropriate choice for
a more sensitive test design (e.g. addressing significant detection rates
$\frac{\geq 69\,\text{correct}}{119\,\text{total}}$ of single treatment
judgment for $\alpha=\beta=0.05$).
%
\NewL Although, here only shown in an ABX comparison scenario and not yet for
musical contents, we cannot fully exclude that
very critical, trained listeners are able to detect constant phase shifts of
well known references even in an non-comparison task as well.
%
Considering this circumstance and the current listening experiment results,
it appears advisable to apply constant phase shift filters in 2.5D and 3D sound
field synthesis applications to perfectly guarantee that
potentially audible phase shift artifacts will not occur.
